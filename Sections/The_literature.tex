\section{Background}\label{Background}
In general terms there are two broad approaches to action-selection for autonomous agents: a learning vs a reasoning approach. The former starts with limited knowledge about the environment, and through sucessive trials and learning from actions and rewards, decides to learn a policy. The latter typically requies more domain knowledge, and reasons about actions before they are made in order to meet some objective/reach a desired state. 
\subsection{Learning vs Reasoning}
While this section is deliberately high-level, it makes a crucial point as to the strengths and weaknesses of data-driven learning vs reasoning approaches. In general, data-driven approaches can perform well, given sufficent time and resources to learn an optimal policy within a stationary environment. This policy often takes a huge amount of time to learn and develop as the agent learns the effects its actions have upon the environment. However, once learned, typically the execution is very swift --- depending on the exact architecture used times vary --- usually a single look-up in a Q table, or a single forward pass through a neural network. They also require very little expert knowledge to impliment, and given a stationary environment, can learn an optimal policy to maximise rewards. 
\newline \newline
In contrast, the reasoning approach requires very few, if any, learnig iterations before performing well. Rather than learning from scratch, reasoning agents typically have a model of the environment (hence the nominer model-based) which they query, testing out the likely effect of each possible action, and subsequent action. Typically, such agents are referred to as planners. These planning agents search for a chain of actions which take them from their initial state to a percieved goal state. As the search space expands exponentially in the number of sequential actions, this search is costly, and hence, at run time, these agents tend to be very slow. 
\newline\newline
In designing a swift agent therefore, one would require the intial performance of a reasoning agent, however the low-latency and swift response of a learning/data-driven agent. 
\newline \newline
One must also address the concept of robustness, and so the consequential qualities of safety (and subsequent trust, or lack thereof): data-driven or learning agents are hard to predict --- should the environment differ from that of their training, performance degrades significantly. 
Further, such are prone to unpredictable and inexplicable action-selection, limiting their applicability from fields which contain any real element of risk. In contrast, reasoning agents are often deemed 'safer'. 
They typically adapt well to changing circumstances, as their internal model already captures some of the environment's dymanics. 
Furthermore, as such agents they investigate the results of each action before they undertake it, given the assumption that their comprehension of the environment dynamics is correct, they act safely and robustly.

\subsection{Opponent Modelling}


\subsection{Reinforcement learning augmented search} \label{RL + Search}
In recent years, the combination of model-based search methods and deep reinforcement learning (via self-play) has achieved super-human performance in a number of benchmark challenges in game playing (adversarial) AI \cite{AlphaGo,Brown2020,Lerer2019}. 
The major difficulties of these domains (Chess, Go, even Poker) is that of a truly enourmous state-space. Research which has acheived noteriety has combined local state-space search with deep neural networks to approximate complex functions. 
\newline \newline
While each of the mentioned papers contribute different novel features, the basic structure of these algorithms remains constant: a sample-based search algorithm augmented with two learning components, commonly referred to as a value network, and a policy network respectively.
\newline \newline
Briefly, these algorithms structure the environment as an extensive form game (a search tree). Nodes represent states, and actions represent transitions from one state to another. Given a simple environment, a planning algorithm can search through the possible states, and select the optimal action in order to maximise rewards. However, given a complex environment, the possible state-space expands exponentially, swiftly rendering a typical search-based solution intractible. To address the issue of an enourmous search space, a \textit{sample-based} search algorithm is employed. 
\newline \newline
A sample based search algorithm constructs a search tree of immediately reachable states (rather than all possible states), given available actions. As there are a huge number of possible sequences of actions, rather than exhaustively searching the entire search-space, a sample-based search tree expands and samples only some branches (possible options), and averages across the expected rewards of immediate actions to select the best action at a given node in the tree at a given time.
\newline \newline
Even with a sampling-based approach, the seach space is often still prohibitively large, and so data-driven learning methods can be used to restrict the search space further to improve performance. Two distinct learning components (deep neural netowrks) are typically employed.  
\newline \newline 
Both networks are trained through a combination of supervised learning and reinforcement learning via self-play (playing oneself). At run-time, the policy network guides the search through the search tree (limiting the branching factor from areas of the state-space which are unlikely/unfavourable). The value network truncates the depth of the search tree by approximating the value (i.e. the expected discounted reward achievable) for a given state. In short, the learning parameters allow for a tractable approximation of vast state-space, while allowing for explicit reasoning over the short-term, or immediate search-space. 
\newline \newline
While arguably the most famous result was DeepMind's Go-playing AI AlphaGo \cite{AlphaGo}, which rather dramatically defeated Lee Sidol, both Chess and Go are perfect information games --- the state of the game is certain and visible at all times. In a development of earlier Poker-playing bots, Brown et al. \cite{Brown2020} tackled the problem of imperfect information games (namely head's up no limit texas hold 'em poker) by adapting the notion of state to a public-belief state. In short, they define a mechanism for transforming an imperfect information game into a continuous state-space perfect information game, where the state contains a probabilistic distribution over all agents beliefs (a public-belief state). Through this transformation, a similar algorithm to Silver et al's AlphaGo can be used, with policy and value networks guiding and truncating search.   
\newline \newline
Crucially, this research approaches the problem of large state-spaces rather more from the side of learning as opposed to reasoning. In contrast, a body of work has tackled partially observable environments with an explicit focus on reasoning: planning.  


\subsection{(Sample-based) Planning}
 Planning is, in essence, a search problem --- an agent queries an internal model which captures the dynamics of the environment to find a sequence of actions to achieve some goal state or maximise a reward. In complex domains, the search-space swiftly becomes so large as to be computationally intractable, hence the recent focus on model-free forms of learning to solve similar problems. However, model-free approaches tend to be unpredictable. As they have no model to reason about the environment, they can exhibit incorrect and even dangerous behaviour in unforeseen scenarios --- typically there is no reasoning process, but simply a mapping of a state to an action. Given this unpredictability, the application of purely model-free approaches to some domains carries risk and is therefore undesirable. Extending planning to complex domains remains an open question and a valuable and viable direction of enquiry. At present, the most promising area appears to be sample-based planning, which requires exploring only some of the search space online (i.e. at runtime).  

\subsubsection{Monte Carlo Tree Search --- an approach to search in a large state-space}
Monte Carlo Tree Search is a sample-based search algorithm which requires a simulation model of the environment to evaluate possible actions and future states in order to converge to an optimal policy online (i.e. re-evaluating from each state). 
Each node represents a possible state, and transitions (edges) are possible (probabilistic) actions. At a given state, an agent traverses a computed tree, selecting actions with a heuristic until reaching a leaf node. At the leaf node, it then randomly selects actions and receives a potential successor state from the transition model. This is referred to as a rollout, and continues to some arbitrary depth of tree depending on available resources. It then evaluates the state and the rewards achieved in the leaf node, and propagates the rewards backwards up the tree. This cycle  (traversal, simulation (rollout) and backpropagation) then continues for a determined period, until a new action is selected, and a new state observed. 
\newline \newline
This sampling-based approach is typically paired with a UCB1 algorithm (upper confidence bounds) which balances exploitable actions with under-explored alternatives, which has shown to converge (eventually) to an optimal policy \cite{Ross2011}. 
\newline \newline
This approach, typically when paired with a strong heuristic and an accurate transition/environment simulator, has shown to be an effective and powerful approach to action-selection in complex environments with large search spaces --- this algorithm forms the basis for most of the recent successes in adversarial AI (as described in the previous Section: Section \ref{RL + Search}) 


\subsubsection{Monte Carlo Search in a partially observable environment --- adopting MCTS to partially observable environments}
Despite the success of a sample-based approach, the extension of planning to uncertain (partially observable) environments still poses problems. 
In such an environemt (due to inperfect or incomplete observations), an agent is (initially) uncertain of the actual state. This casuses considerable complexity, and has ramifications for the time and space requirements for computing an optimal policy.
In an $n$-stateful environment, an agent must compute a distribution over $n$ states representing its belief over the true state of the environment. 
It must compute this in addition to the generic planning complexities of all possible actions, transitions and resulting states --- hence unless facing a very trivial problem in an exceedingly simple environment, the exponential complexity of partially observable environments tend to render planners inapplicable. 
\newline \newline
Silver and Veness  \cite{Silver2010} combined a sampling-based approach to belief updating (particle-filtering) with a monte-carlo tree search style algorithm which allows for partially observable planning. Among a number of alterations, each node in the tree is based on an observation history rather than a state, reflecting the agents belief over the state. In effect, Monte Carlo methods are used in a bi-directional manner: forwards to evaluate action selection; backwards to update belief over states. 
\newline \newline
In small state spaces, the belief-state (i.e. distribution over the possible state) can be perfectly calculated by applying Bayes rule --- in large state spaces this can be computationally demanding, and a compact representation of the transition model (in terms of likelhihoods) might not be available. To address this problem Silver and Veness contributed an algorithm, Partially Observable Monte Carlo Planning (POMCP), which uses a particle-filter: generating a number of small unweighted particles, each representing a possible state, and evaluating them based on expected vs received observations. They use this sampling-based approach to update the belief about the likely history at each time-step, and iteratively converge onto the true state.
\newline \newline
As with typical Monte Carlo methods, the sampling approach greatly limits the search space as only (likely) reachable states are evaluated, and the belief over states can be efficiently computed. In short, sampling based methods have shown to allow for swift approximation over complex belief spaces.

\subsubsection{Improving on POMCP --- updating the environment simulation}  
Despite the theoretical success of POMCP, belief-based planning still poses challenges. 
While known for its efficacy in dealing with large state-spaces, MCTS is limited by the accuracy of the transition model. Simply put, if the model of the environment provided to the planner is not sufficiently expressive as to capture the environment dynamics, despite efficient sampling, the performance of an agent will be poor. 
\newline \newline
In recent years, there have been two notable works which have augmented belief-based planners with a learning capacity in order to overcome an incorrectly specified model. 
\newline \newline
Katt et al \cite{Katt2017} augmented the notation of a POMDP to take account of a number of extra features, to allow for a Bayesian updating of the environment model --- in effect the transition dynamics were updated as experience increased in a Bayesian way, allowing for more realistic rollouts and better action selection. In short, they incorportated a learning element into the model of the environment to allow for an incorrectly or incomplete black-box environment simulation. 
\newline \newline
With a similar aim, Hayashi et al \cite{Hayashi_et_al2020} augmented a POMCP planner with a deep-recurrent neural network as a mechanism for particle reinvigoration (suggesting possible states to evaluate), which suggested better candidate states (particles) to be computed allowing noisy or incorrectly specified environments to be used without resulting in poor agent performance. They phrased the environment simulator as a black-box with various parameters which required fitting. This fitting took place as experience grew, meaning that the black-box simulator could be tuned online, allowing the world-model to be iteratively updated. Crucially, they applied this to an opponent modelling task, the level-based foraging domain \cite{Papoudakis2020,Barrett2015}, in which agent types were parameters within the black-box simulator. 

\subsubsection{Improving MCTS --- information theoretic rewards}
Finally, Fischer and Tas augmented a MCTS style of planning with information theoretic rewards in a continuous domain. This means in practice increasing an internal reward for reaching states where the agent received more information about the environment. The basic idea behind this is that often as an agent's understanding of a domain increases, the convergence to optimal action selection is swifter. While this was tested in a single-agent domain, the concept of augmenting a sampling-based search procedure with information-based rewards showed promise \cite{Fischer2020}. Despite the domain being marginally different from that of opponent modelling, the concept of explicitly rewarding an agent for looking for information has parallels in opponent exploration. Ultimately, the concept of deliberately manipulating a game state to learn more about an opponent is worthy of investigation, and seems at present somewhat under-explored.

\subsubsection{Incoporating agent models into MCTS}
As mentioned previously, a MCTS involves several stages: 1) traversing a tree via a best-first (heuristic) until reaching a leaf node; 2) Upon reaching a leaf node, it performs a rollout, which involves selecting an action, and sampling an environment simulator for a possible next state and expected reward. It continues this rollout to an arbitrary depth. 3) Finally, having reached a maximum deapth, the reward is propagated back up the tree to the root node.
\newline \newline 
In a multi-agent environment, the transition function responsible for mapping one state to another depends on a joint actions taken by all agents. Hence, in a rollout, the simulator must assume a joint action (i.e. infer/suggest likely actions taken by \textit{other} agents). In this way, agent models are integrated into a MCTS via the simulation phase. In simple terms, they are implicitly included in the environment model. In both \cite{Hayashi_et_al2020} and \cite{Albrecht_stone_2019} agent models are computed and viewed as parameters to be tuned in this environment, thus providing an interesting middle ground between a `global' approach, in which an agent simply learns a single generative model of  the environment from scratch (no opponent models), and a typical opponent modelling approach, which requires distict agent models and environment models.
\newline \newline
This method also has benefits as it translates the problem of opponent modelling into one of state-space search, allowing for the incorporation of policy/value netowrks and/or sampling-based planners to address complexities of opponent modelling, which often result in large (probabilistic) state-spaces.