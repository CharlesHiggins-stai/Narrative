\section{Formalisation} \label{MDP/POMDP}
\subsection{Partially observable Markov Decision Process POMDP}
The environment can be phrased as a partially observable Markov decision process (POMDP). 
A POMDP is defined as a tuple: 
\begin{equation}
    \langle S, A, O, P, Z, R, \gamma, b_0 \rangle
\end{equation}

\noindent$S$ is the set of states, $A$ the set of actions, $O$ the set of observations; 
\newline \newline  
$P = Pr(s_{t+1}|s_t, a_t)$ is a stochastic transition function, mapping a state and action to a new state. \newline \newline 
 $Z = Pr(o_{t+1}|s_{t+1},a_{t})$ is the stochastic observation function --- observations are not certain, hence the it is defined as the probability of observing a certain observation having taken an action in a given state. 
 \newline \newline
 $R = R(s_t, a_t)$ is the reward function;
 \newline \newline
 $\gamma \in [0,1]$ is the discount factor (discounting future rewards);
 \newline \newline 
 $b_0$ is the belief about the initial state.
\newline \newline 
In every timestep $t$, an agent receives a reward ($r_t$) and an observation ($o_t$) and takes/selects an action. The agent aims to derive a policy $\pi$ which maximises its total discounted reward given its beliefs in a certain state. 
\newline \newline
Crucial to efficient planning and action selection for a model-based agent is minimising or eliminate the mismatch between the agent's belief/encoded model of the environment and the dynamics of the real environment. This is challenging given that observations are not perfect and only display part of the environment, and the transition between each state is not certain, and doesn't solely depend on a single agent's action.
\newline \newline
To simplify, let the state-transition function be interpreted as $s_{t+1} \sim f(s_t,a_t)$, and the observation function be $o_{t+1} \sim g(s_{t+1}, a_t)$. 
\newline \newline
Given that the state can be changed by other agents, one can view the environment as having a joint action-space between the actions of an agent and all other agents present. Hence, with the assumption that another agent is present, one can then reform the state-transition function and observation functions to take into account the joint actions of all other agents and the environment dynamics, which shall be represented by $\Theta$: 
\begin{equation}
    s_{t+1} \sim f(s_t, a_t|\Theta) 
\end{equation}
\begin{equation}
    o_{t+1} \sim g(s_{t+1}, a_t|\Theta) 
\end{equation}
\newline \newline
Opponent modelling can be seen as an attempt to improve this model $\Theta$, by explicitly modelling an opponent, or group of opponents. 
The quality of an agent's internal model, and therefore the quality of the overall policy derived, is dictated by the quality of the state-transition and observation functions. 
\newline \newline
One can view $\Theta$ as parameters to be tuned for an encapsulated (internal) environment model, which when fed to a black box environment simulation (the function $ f(s_t, a_t|\Theta$) provides a projected next state.
\newline \newline
By way of an example, an agent would enter an environment with a belief over the likely parameter(s) of $\Theta$. As interactions continue, an agent can update the value(s) of $\Theta$ based on observations: the values of $\Theta$ are not immediately obvious however they can be inferred from observations/prior knowledge and some form of reasoning. Hence, the belief of an agent at time t can be seen as a joint distribution over:
\begin{equation}
    b_t(s,\Theta) = Pr(s_t,\Theta | h_t)
\end{equation}
where $h_t$ is the history of observation, reward, action triples ($h_t = [(a_0, o_0,r_0), (a_1, o_1, r_1) ... (a_{t-1}, o_{t-1}, r_{t-1})]$) taken to reach the state $s_t$. In other words, the opponent model is included in an uncertain state-representation. Hence, in reasoning over the state, the agent also reasons over possible opponent models given an action history. 

\subsection{Parameterisation of $\Theta$} \label{IncludingAgentModels}
The previous section provides a general framework of simulation-based action-selection which makes use of an internal agent model to better inform an observation and state-transition function. This serves to show how an opponent model might fit into the action-selection pipeline.
\footnote{The definition of $\Theta$, and methods to accurately estimate $\Theta$, where theta refers to an number of opponent models in an open environment is completed by a simulation based component known as a particle simulator (this is explained further in Section \ref{ParticleFiter})}.
\newline\newline
In the simplest form, $\Theta$ refers to a vector containing agent models: ${\theta_0, \theta_1,...,\theta_n} \in \Theta$. 
\subsubsection{Type-based reasoning}
Each $\theta_i \in \Theta$ is an agent present in the environment.    
A popular and swift method of opponent modelling is type-based reasoning. Rather than building an assumed policy from scratch at run time, a modelling agent assigns a pre-specified agent model to an observed agent. This has often been supplied by the user, or (recently) has been learned over repeated interactions with other (similar) agents. 
This leads to a (potentially) large number of agent \textit{types}: $\phi \in \Phi$.
\newline \newline
Hence, the parameter of $\phi$ in $\theta_i$ refers to agent model $\theta_i$ as being of type $\phi$. 
\newline \newline
Hence, in terms of the belief updating, one can rewrite the belief over state and model parameters, given an observation history to the belief over state and type of model, given an observation history. 
\begin{equation}
    b_t(s,\Theta) = Pr(s_t,\theta_i = \phi | h_t)
\end{equation}




